\documentclass[12pt]{article}

\author{Tom Hubrecht}
\title{Transmission d'informations entre la Terre et les sondes spatiales}
\date{Juin 2019}

\usepackage[utf8]{inputenc}
\usepackage[left=2cm, right=2cm, top=2cm]{geometry}
\usepackage{lmodern}
\usepackage{graphicx}
\usepackage{tikz}
\usepackage{amsmath}

\usetikzlibrary{shapes,arrows}

\graphicspath{{./Images/}}

\begin{document}

\maketitle

\section{Introduction}
L'exploration spatiale est un domaine majeur et actuel de la recherche scientifique, les sondes envoyées ne pouvant souvent pas revenir sur Terre, il est crucial de recevoir les informations récoltées, ce qui me semble être un des aspects les plus intéressants de ce domaine. Les liaisons entre les sondes spatiales et la Terre établissent le transport d'instructions de la Terre vers la sonde ainsi que la transmission en retour des données obtenues par la sonde.


\section{Probl\'ematique}
La r\'eception correcte des informations transmises par une sonde spatiale est conditionn\'ee par le codage des donn\'ees de telle sorte que l'on puisse n\'egliger le bruit occasion\'e par la transmission du message \`a travers l'espace. Puisque plusieurs m\'ethodes de codage ainsi que diff\'erentes mani\`eres de d\'ecoder les messages re\c{c}us sont en concurrence, il faut donc \'etudier le fonctionnement des ces algorithmes ainsi que de les impl\'ementer pour tester leurs performances dans une simulation.

\section{Choix des codes utilis\'es}



\end{document}